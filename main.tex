\documentclass[10pt,A4paper]{article}

\usepackage{acronym}
\usepackage{amsmath,amssymb}
\usepackage[aboveskip=1pt,labelfont=bf,labelsep=period,justification=raggedright,singlelinecheck=off]{caption}
\usepackage{changepage}
\usepackage{cite}
\usepackage{nameref,hyperref}
\usepackage[right]{lineno}
\usepackage[nopatch=eqnum]{microtype}

\usepackage{xcolor}

\usepackage[outputdir=out]{minted}
\definecolor{bg}{HTML}{282828} % fro https://github.com/kevinsawicki/monokai
\setminted{autogobble=true,bgcolor=bg,linenos=true,style=monokai}

\bibliographystyle{abbrv}

% Header and Footer with logo
\usepackage{lastpage,fancyhdr,graphicx}
\usepackage{epstopdf}
\pagestyle{fancy}
\fancyhf{}
\rfoot{\thepage/\pageref{LastPage}}
\renewcommand{\headrulewidth}{0pt}
\renewcommand{\footrule}{\hrule height 2pt \vspace{2mm}}
\lfoot{\today}

\newcommand{\etal}{{\textit{et al.}}}
\newcommand{\mbx}{\mathbf{x}}
\newcommand{\mbu}{\mathbf{u}}
\newcommand{\mbp}{\mathbf{p}}
\newcommand{\mby}{\mathbf{y}}
\providecommand{\keywords}[1]{\textbf{Keywords } #1}


% Create acronyms to use in text
\newacro{ode}[ODE]{Ordinary Differential Equation}
\newacro{oed}[OED]{optimal experimental design}


\begin{document}
% ########################################################################
% ########################################################################
\vspace*{0.2in}
% Title must be 250 characters or less.
\begin{flushleft}
{\Large
\textbf\newline{Model Calibration with Process Variables depending on Environmental Parameters}}
\newline
\\
Polina Gaindrik\textsuperscript{1,2,3},
Jonas Pleyer\textsuperscript{1,2},
Christian Fleck\textsuperscript{1,2,3}
\\
\bigskip
\textbf{1} \href{https://www.fdm.uni-freiburg.de/spatsysbio}{University of Freiburg}\\
\textbf{2} \href{https://www.fdm.uni-freiburg.de/spatsysbio}{Freiburg Center for Data Analysis and Modeling}\\
\textbf{3} \href{https://tsenso.com/en/}{Tsenso}\\
\bigskip

\end{flushleft}
% ########################################################################
% ########################################################################
\section*{Abstract}
\linenumbers
Parameter estimation is a significant part of the model building that is always accompanied by quantification uncertainty.
Experimental design allows optimization of the observation conditions, reduces the uncertainty of the parameter estimation, and descreases the number of measurements. 
In this chapter, we provide instructions for finding the optimal experimental design using the (Baranyi and Roberts ?) growth model as an example. 
The guidelines are supposed to clarify the procedure and make it more accessible for a general reader of Systems Biology.
As an example, we provide a Python-written software that can be used for simple systems (??).

\keywords{Optimal experimental design, Parameter estimation, Fisher Information matrix, Identifiability, Uncertainty}

%
%
%
% ########################################################################
% ########################################################################
\section*{Introduction}
% \begin{figure}[h]
% 	\inputminted[linenos,firstline=57,lastline=79]{python}{../model-design-fischer-information-matrix/pool_model.py}
% 	\caption{Sample code written in Python~\cite{rossumPythonLanguageReference2010}.}
% \end{figure}
%
\begin{enumerate}
	\item Why do we want parameter estimation and experimental design?
	\item Flow-Chart Experimental Design
	\item Citations to common methods
\end{enumerate}


Mathematical modeling is a widely used tool to describe, understand and predict further behavior of living systems.
In particular, in the field of Predictive Biology, one can find a large variety of works that dwell on building models of different levels of complexity controlled by model parameters, e.g., to describe bacteria growthBased on a chosen model structure, these parameters can be estimated from the gathered experimental data.
Taking into account that the real experimental data always contains measurement noise, the parameter estimates can be only provided with some uncertainty. 
To decrease the error of the parameter values, not only enough experimental data should be gathered but the quality of this data is also pretty sufficient.
That rises quite an important question of finding the \ac{oed} where optimized experimental conditions and/or times allow one to reduce the number of measurements without loss of information thus sparing an effort of experimenters \cite{derlindenImpactExperimentDesign2013, balsa-cantoe.bangaj.r.COMPUTINGOPTIMALDYNAMIC2008}. 
In general, \ac{oed} can be used not only for the parameter estimation of the model but also for model discrimination to choose between different model structures \cite{kreutzSystemsBiology2009, stamatiOptimalExperimentalDesign2016}.
However, in this chapter, we would like to cover more deeply the methods for improving the parameter estimation of the model.
According to this, using tools of Statistics and specifically the Fisher information, several criteria were developed to assess data quality. 
The review of these different possible criteria the reader can find for example in works \cite{atkinsonDevelopmentsDesignExperiments1982, franceschiniModelbasedDesignExperiments2008}.
Depending on the goal, a researcher faces an important choice which among these sometimes contradicting each other criteria is the most suitable one for a particular case.
On the other hand, using the multi-objective approach, some attempts were made to improve the experimental design by combining these criteria \cite{telenOptimalExperimentDesign2012, logistRobustMultiobjectiveOptimal2011}.
If we are talking about predictive microbiology, one of the promising applications of the \ac{oed} is to optimize the temperature profile with the time of the experiment as it was done by García \etal \cite{garciaQualityShelflifePrediction2015} and Versyck \etal \cite{versyckIntroducingOptimal1999}.
Or, for example, Balsa-Canto \etal compared equidistant and optimized measurement times for such bacteria growth models as Baranyi and Roberts model and Bigelow model and have shown improvement in the latter case \cite{balsa-cantoe.bangaj.r.COMPUTINGOPTIMALDYNAMIC2008}.

\begin{figure}[H]
    \centering
    \includegraphics[scale=0.3]{Figures/scheme.png}
    \caption{The iterative process of model optimization for parameter estimation.}
    \label{fig:expdesign_scheme}
\end{figure}
In the beginning, we would like to briefly describe the whole iterative process of model building (see Fig. \ref{fig:expdesign_scheme}).
First of all, based on the literature review or some prior data from pilot experiments the first parameter estimation of the chosen model structure should be done.
The obtained values can be applied to propose the first optimal experimental design accounting for different constraints, e.g., the lab limitations, human resources, etc. 
Depending on availability, either real or numerical experiment should be conducted based on this design to gather measurement or in-silico data. 
This new data can be used for the new parameter estimations values with corresponding uncertainties.
After this, using new parameter values, the process can be repeated several times to increase the precision of the parameter estimates till the desired accuracy is achieved.







%
%
%
% ########################################################################
% ########################################################################
\section*{Materials and Methods}
This tutorial expects a working installation of the popular scripting language Python $\geq3.7$~\cite{rossumPythonLanguageReference2010}.
For installation instructions, please refer to the website \href{https://www.python.org/downloads/}{python.org}.
We have developed \mintinline[bgcolor=white,style=emacs]{bash}{FisInMa}, which is a set of tools to calculate the Sensitivity and Fischer Information Matrix and do parameter space exploration in order to find optimal results.
Users can obtain it by installing \mintinline[bgcolor=white,style=emacs]{bash}{FisInMa} from \href{https://pypi.org/project/FisInMa/0.0.1/}{pypi.org}.
The python website has guides for installing packages \href{https://packaging.python.org/en/latest/tutorials/installing-packages/}{packaging.python.org/}.
Most Unix-Based systems (eg. GNU/Linux) and Mac-OS can use \mintinline[bgcolor=white,style=emacs]{bash}{pip}
\begin{minted}[bgcolor=white]{bash}
pip install FisInMa
\end{minted}
or \mintinline[bgcolor=white,style=emacs]{bash}{conda} to install the desired package.
\begin{minted}[bgcolor=white]{bash}
conda install FisInMa
\end{minted}
%
%
% ########################################################################
\subsection*{Model Formulation}
\subsubsection*{Theory}
\begin{enumerate}
    \item Define ODE, Jacobian
    \item Which parameters are present?
    \item Differnce between Q-Values, P-Values, Const
\end{enumerate}

The first important step a reader should do to conduct their Experimental Design is to define a model structure.
In this chapter, we restrict ourselves to a quite popular case in Predictive Microbiology and assume that biological system is described by a system of \acp{ode}
\begin{equation}
    \begin{cases}
    \dot \mbx (t) = f(\mbx, t, \mbu, \mbp) \\
    \mbx (0) = \mbx_0
    \end{cases}.
\end{equation}
Here $\mbx = (x_1, x_2, ..., x_n)$ is a vector of state variables of the system with initial condition $\mbx_0$, $t$ is time, $\mbu$ is a vector of an externally controlled inputs and $\mbp$ are the parameters of the system that be want to estimate. 
Another important choice to be done is to define the observables of the system $\mby$, which quantities and at which times $t_i$ should be measured.
\begin{equation}
    \mby (t_i) = g(\mbx (t_i), t, \mbu, \mbp) + \epsilon (t_i),
\end{equation}
where the function $g$ describes the model output and $\epsilon$ is the measurement noise. 
The observational noise is often assumed to be an independently distributed Gaussian noise.

Here, to demonstrate how Experimental Design works, we will refer to a widely used Baranyi and Roberts model (1994) that can be used, for example, to describe bacteria growth.
The model is introduced by two state variables $\mbx = (x_1, x_2)$, where $x_1(t)$ denotes the cell concentration of a bacterial population at the time $t$ and $x_2(t)$ defines a physiological state of the cells, the process of adjustment (lag-phase):
\begin{equation}
    \begin{cases}
    \dot x_1(t) = \frac{x_2(t)}{x_2(t) + 1} \mu^\text{max} \big(1 - \frac{x_1(t)}{x_1^\text{max}}\big) x(t)  = f_1 \\
    \dot x_2(t) = \mu^\text{max}  x_2(t) = f_2
    \end{cases}.
    \label{eq:ode_BaranyiRoberts}
\end{equation}
Here $\mu^\text{max}$ determines the maximum growth rate, and $x_1^\text{max}$ is bacteria concentration at the saturation. 
To account for the influence of the temperature on the activity of the model, we will use the 'square root' or Ratkowsky-type model for the maximum growth rate
\begin{equation}
    \sqrt{\mu^\text{max}} = b (T - T_\text{min}).
    \label{eq:RatkowskyModel}
\end{equation} 
As an observable, it is pretty common to measure the bacteria count $x_1$ or the logarithm of this value. 
For simplicity, we would consider the prior case $y(t_i) = x_1(t_i)$.
The system is fully defined by equations (\ref{eq:ode_BaranyiRoberts}), (\ref{eq:RatkowskyModel}).
Here $x_1^\text{max}, b, T_\text{min}$ are the parameters that we estimate using observational data $y$ at measurement times $t_i$, and temperature $T$ is an input of the system.
Based on this model, we would like to optimize the choice of measurement times as well as temperatures (inputs) of the system to find the \acl{oed}.


\subsubsection*{Code}
\begin{enumerate}
    \item Write ODE, Jacobian functions in python with Q, P, Const\\
    \mintinline[bgcolor=white,style=emacs]{python}{def ode_func(y, t, Q, P, Const):}\\
    \mintinline[bgcolor=white,style=emacs]{python}{def jacobian(y, t, Q, P, Const):}
    \item Define initial values \mintinline[bgcolor=white,style=emacs]{python}{(y0, t0)}
\end{enumerate}
%
% ########################################################################
\subsection*{Parameter Estimation}
\subsubsection*{Theory}
\begin{enumerate}
    \item Log-Likelihood Function
    \item Maximum-Likelihood Estimators
    \item Likelihood Function for Gaussian Noise
\end{enumerate}

One of the requirements for conducting an Experimental Design is a provision of parameter values by a user.
Due to this, after the model was defined, the first parameter values should be chosen from the literature or estimated from previously gathered experimental data by maximizing the likelihood function.
By definition, the likelihood function $L(\mbp)$ is the probability of observing a data $\mby$ in case of a certain parameters $\mbp$:
\begin{equation}
    L(\mbp) = p(\mby|\mbp).
    \label{eq:likelihood_definition}
\end{equation}
In the case of the Gaussian white measurement noise with zero mean and variance $\sigma^2$: $\epsilon(t_i) \propto N(0, \sigma_i^2)$, it is convenient to maximize the logarithm of the likelihood function (log-likelihood) instead.
This is likewise simply minimizing the sum of the squared difference between the measured data and model output:
\begin{equation}
    \ln L(\mbp) \propto - \sum_{i}\frac{ \big(\mathbf{g}^{i}(\theta) - \mby^{i}\big)^2}{2 \sigma_{i}^2}.
    \label{eq:likelihood_Gaussian}
\end{equation}
Here $\mby^{i}$ is the $i$-th value of observational data, and $\mathbf{g}^{i}(\theta)$ is a corresponding to it model output.

\subsubsection*{Code}
\begin{enumerate}
    \item \mintinline[bgcolor=white,style=emacs]{python}{def likelihood_function(fitted_parameter, measurement_data)}
    \item \mintinline[bgcolor=white,style=emacs]{python}{scipy.optimize.minimize}
\end{enumerate}
%
% ########################################################################
\subsection*{Experimental Design}
\subsubsection*{Theory}
\begin{enumerate}
    \item Fischer, Sensitivity Matrix
    \item Observables: Determinant, Eigenvalues, etc.
\end{enumerate}
\subsubsection*{Code}
\begin{enumerate}
    \item How does the user calculate the Fischer Observable?
    \begin{minted}{python}
        def calculate_Fischer_observable(
            parameter_combinations,
            sensitivity_ode_function,
            y0_t0,
            jacobi,
            observable):
    \end{minted}
    \item How do we optimize for best results?
    \begin{minted}{python}
        def optimize_parameters(
            parameter_combinations,
            observable):
    \end{minted}
\end{enumerate}
%
%
%
% ########################################################################
% ########################################################################
\section*{Conclusion}
\begin{enumerate}
    \item Plots: Optimized time points
    \item Observable vs number of measurements
    \item Performance (scaling, limitations, etc.)
    \item 
\end{enumerate}
%
%
%
% ########################################################################
% ########################################################################
\section*{Supporting information}
%
%
%
\nolinenumbers
% ########################################################################
% ########################################################################
\bibliography{predictive-microbiology-software}

\end{document}
