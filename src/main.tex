\documentclass[10pt,A4paper]{article}

\usepackage{amsmath,amssymb}
\usepackage[aboveskip=1pt,labelfont=bf,labelsep=period,justification=raggedright,singlelinecheck=off]{caption}
\usepackage{changepage}
\usepackage{cite}
\usepackage{nameref,hyperref}
\usepackage[right]{lineno}
\usepackage[nopatch=eqnum]{microtype}

\usepackage{xcolor}

\usepackage[outputdir=../out]{minted}
\definecolor{bg}{HTML}{282828} % from https://github.com/kevinsawicki/monokai
\setminted{autogobble=true,bgcolor=bg,linenos=true,style=monokai}

\bibliographystyle{abbrv}

% Header and Footer with logo
\usepackage{lastpage,fancyhdr,graphicx}
\usepackage{epstopdf}
\pagestyle{fancy}
\fancyhf{}
\rfoot{\thepage/\pageref{LastPage}}
\renewcommand{\headrulewidth}{0pt}
\renewcommand{\footrule}{\hrule height 2pt \vspace{2mm}}
\lfoot{\today}


\begin{document}
% ########################################################################
% ########################################################################
\vspace*{0.2in}
% Title must be 250 characters or less.
\begin{flushleft}
{\Large
\textbf\newline{Model Calibration with Process Variables depending on Environmental Parameters}}
\newline
\\
Polina Gaindrik\textsuperscript{1,2,3},
Jonas Pleyer\textsuperscript{1,2},
Christian Fleck\textsuperscript{1,2,3}
\\
\bigskip
\textbf{1} \href{https://www.fdm.uni-freiburg.de/spatsysbio}{University of Freiburg}\\
\textbf{2} \href{https://www.fdm.uni-freiburg.de/spatsysbio}{Freiburg Center for Data Analysis and Modeling}\\
\textbf{3} \href{https://tsenso.com/en/}{Tsenso}\\
\bigskip

\end{flushleft}
% ########################################################################
% ########################################################################
\section*{Abstract}
\linenumbers
%
%
%
% ########################################################################
% ########################################################################
\section*{Introduction}
% \begin{figure}[h]
% 	\inputminted[linenos,firstline=57,lastline=79]{python}{../model-design-fischer-information-matrix/pool_model.py}
% 	\caption{Sample code written in Python~\cite{rossumPythonLanguageReference2010}.}
% \end{figure}
%
\begin{enumerate}
	\item Why do we want parameter estimation and experimental design?
	\item Flow-Chart Experimental Design
	\item Citations to common methods
\end{enumerate}
%
%
%
% ########################################################################
% ########################################################################
\section*{Materials and Methods}
%
%
% ########################################################################
\subsection*{Model Formulation}
\subsubsection*{Theory}
\begin{enumerate}
    \item Define ODE, Jacobian
    \item Which parameters are present?
    \item Differnce between Q-Values, P-Values, Const
\end{enumerate}
\subsubsection*{Code}
\begin{enumerate}
    \item Write ODE, Jacobian functions in python with Q, P, Const\\
    \mintinline[bgcolor=white,style=emacs]{python}{def ode_func(y, t, Q, P, Const):}\\
    \mintinline[bgcolor=white,style=emacs]{python}{def jacobian(y, t, Q, P, Const):}
    \item Define initial values \mintinline[bgcolor=white,style=emacs]{python}{(y0, t0)}
\end{enumerate}
%
% ########################################################################
\subsection*{Parameter Estimation}
\subsubsection*{Theory}
\begin{enumerate}
    \item Log-Likelihood Function
    \item Maximum-Likelihood Estimators
    \item Likelihood Function for Gaussian Noise
\end{enumerate}
\subsubsection*{Code}
\begin{enumerate}
    \item \mintinline[bgcolor=white,style=emacs]{python}{def likelihood_function(fitted_parameter, measurement_data)}
    \item \mintinline[bgcolor=white,style=emacs]{python}{scipy.optimize.minimize}
\end{enumerate}
%
% ########################################################################
\subsection*{Experimental Design}
\subsubsection*{Theory}
\begin{enumerate}
    \item Fischer, Sensitivity Matrix
    \item Observables: Determinant, Eigenvalues, etc.
\end{enumerate}
\subsubsection*{Code}
\begin{enumerate}
    \item How does the user calculate the Fischer Observable?
    \begin{minted}{python}
        def calculate_Fischer_observable(
            parameter_combinations,
            sensitivity_ode_function,
            y0_t0,
            jacobi,
            observable):
    \end{minted}
    \item How do we optimize for best results?
    \begin{minted}{python}
        def optimize_parameters(
            parameter_combinations,
            observable):
    \end{minted}
\end{enumerate}
%
%
%
% ########################################################################
% ########################################################################
\section*{Conclusion}
\begin{enumerate}
    \item Plots: Optimized time points
    \item Observable vs number of measurements
    \item Performance (scaling, limitations, etc.)
    \item 
\end{enumerate}
%
%
%
% ########################################################################
% ########################################################################
\section*{Supporting information}
%
%
%
\nolinenumbers
% ########################################################################
% ########################################################################
\bibliography{predictive-microbiology-software}

\end{document}
